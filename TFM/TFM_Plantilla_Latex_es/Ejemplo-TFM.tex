\documentclass[12pt,a4paper,twoside]{book}
\usepackage{graphicx}
\usepackage{setspace}	%double spacing for text, single for captions, footnotes, etc.
%\usepackage{hypernat} 	%substitut de cite que permet fer hyperlinks
\usepackage{natbib}		% substituye a 'hypernat' que funciona en Windows.
\usepackage[spanish]{babel}
\usepackage[utf8]{inputenc}
\usepackage{color}
\usepackage{hhline} 		% extended styles for tables
\usepackage{multirow}
\usepackage{subfigure}
\usepackage{acronym}
\usepackage{hyperref}
\usepackage{amsmath,amsmath,amssymb} 
\usepackage{fancyhdr}
\usepackage{epsfig, amsmath}
\usepackage{algorithm}
\usepackage{algorithmic}

% general settings
\hypersetup{
	linktocpage=true,
	colorlinks=true,
	linkcolor=blue,
	citecolor=blue,
}
\definecolor{Hgray}{gray}{0.6}

\newenvironment{definition}[1][Definition]{\begin{trivlist}
\item[\hskip \labelsep {\bfseries #1}]}{\end{trivlist}}

\setlength{\topmargin}{0cm}
\setlength{\textheight}{23cm}
\setlength{\textwidth}{17cm}
\setlength{\oddsidemargin}{0cm}
\setlength{\evensidemargin}{0cm}
\setlength{\headheight}{1cm}

% indica que las 'sub-sub-sections' sean numeradas y aparezcan en el indice
\setcounter{secnumdepth}{3}
\setcounter{tocdepth}{2}

% settings for code
\renewcommand{\algorithmicrequire}{\textbf{Entrada: }}
\renewcommand{\algorithmicensure}{\textbf{Salida: }}

%%%%%%%%%%%%
% DOCUMENT %
%%%%%%%%%%%%
\begin{document}

% portada
\newpage
\thispagestyle{empty}

\baselineskip 2em

%\vspace*{1cm}

\centerline{\includegraphics[width=0.6\textwidth]{images/UOC-logo}}
\begin{center}
\textsc{Universitat Oberta de Catalunya (UOC) \\
 Máster Universitario en Ciencia de Datos (\textit{Data Science})\\}

%\centerline {\pic{UOC}{4cm}}

\vspace*{1.5cm}

\textsc{\Large TRABAJO FINAL DE MÁSTER}

\vspace*{0.5cm}

\textsc{\large Área: YYY}


%\textbf{\Huge VirtualTechLab Model: }

\vspace*{2.0cm}

\textbf{\Large xxx título del trabajo xxx}

\textbf{\large xxx subtítulo (en caso de existir) xxx}

\vspace{2.5cm}
\baselineskip 1em

\baselineskip 2em
-----------------------------------------------------------------------------\\
Autor:      Nombre completo del estudiante\\
Tutor:      Nombre del colaborador/a docente\\
Profesor:   Nombre del profesor responsable del área de TF\\
-----------------------------------------------------------------------------\\
\vspace*{1.5cm}
Barcelona, \today

\end{center}

\newpage
\pagestyle{empty}
\hfill

\newpage
% abstract
\pagenumbering{roman} 
\setcounter{page}{1} 
\pagestyle{plain}

%%%%%%%%%%%%%%%%
%%% CREDITOS %%%
%%%%%%%%%%%%%%%%
\chapter*{Créditos/Copyright}

Una página con la especificación de créditos/copyright para el proyecto (ya sea aplicación por un lado y documentación por el otro, o unificadamente), así como la del uso de marcas, productos o servicios de terceros (incluidos códigos fuente). Si una persona diferente al autor colaboró en el proyecto, tiene que quedar explicitada su identidad y qué hizo.

A continuación se ejemplifica el caso más habitual, aunque se puede modificar por cualquier otra alternativa:

\vspace{1cm}

\begin{figure}[ht]
    \centering
	\includegraphics[scale=1]{images/license.png}
\end{figure}

Esta obra está sujeta a una licencia de Reconocimiento -  NoComercial - SinObraDerivada

\href{https://creativecommons.org/licenses/by-nc-nd/3.0/es/}{3.0 España de CreativeCommons}.

%%%%%%%%%%%%%
%%% FICHA %%%
%%%%%%%%%%%%%
\chapter*{FICHA DEL TRABAJO FINAL}

\begin{table}[ht]
	\centering{}
	\renewcommand{\arraystretch}{2}
	\begin{tabular}{r | l}
		\hline
		Título del trabajo: & Descriptivo del trabajo\\
		\hline
        Nombre del autor: & Nombre y dos apellidos\\
		\hline
        Nombre del colaborador/a docente: & Nombre y dos apellidos\\
		\hline
        Nombre del PRA: & Nombre y dos apellidos\\
		\hline
        Fecha de entrega (mm/aaaa): & MM/AAAA\\
		\hline
        Titulación o programa: & Plan de estudios\\
		\hline
        Área del Trabajo Final: & El nombre de la asignatura de TF\\
		\hline
        Idioma del trabajo: & Catalán, español o inglés\\
		\hline
        Palabras clave & Máximo 3 palabras clave\\
		\hline
	\end{tabular}
\end{table}

%%%%%%%%%%%%%%%%%%%
%%% DEDICATORIA %%%
%%%%%%%%%%%%%%%%%%%
\chapter*{Dedicatoria/Cita}

Breves palabras de dedicatoria y/o una cita.

%%%%%%%%%%%%%%%%%%%
%%% Agradecimientos %%%
%%%%%%%%%%%%%%%%%%%
\chapter*{Agradecimientos}

Si se considera oportuno, mencionar a las personas, empresas o instituciones que hayan contribuido en la realización de este proyecto.

%%%%%%%%%%%%%%%%
%%% RESUMEN  %%%
%%%%%%%%%%%%%%%%
\chapter*{Abstract}
\addcontentsline{toc}{chapter}{Abstract}

\onehalfspacing

Texto con la síntesis del proyecto, esto es, un texto en el cual se explica de manera concisa la definición del proyecto/problema abordado, sus objetivos/métodos de resolución, y los resultados y conclusiones (no puede ser una lista, sino un texto continuo redactado de manera estructurada). Si es necesario poner una referencia en este texto, ésta será anotada a pie de la misma página. En este apartado se puede usar un lenguaje más literario y coloquial que para el resto del documento.

El Abstract se escribirá por duplicado. Una de las versiones tiene que ser \textbf{obligatoriamente en inglés}. La otra versión tiene que estar escrita en catalán o español. En caso de no escribir el resto del documento en inglés, será necesario escribir la segunda versión del Abstract en el idioma utilizado para el resto de la memoria. La palabra Abstract se cambiará por ``\textbf{Resum}'' o ``\textbf{Resumen}'' en la versión catalana y española, respectivamente. 

Extensión recomendada: 250 palabras máximo.

Como escribir un buen Abstract (en inglés):

\href{http://www.ece.cmu.edu/~koopman/essays/abstract.html}{http://www.ece.cmu.edu/~koopman/essays/abstract.html}

\vspace{1.5cm}

\textbf{Palabras clave}: Keywords del trabajo separadas por comas. Por ejemplo para este documento podrían ser Modelo, Pauta, Plantilla, Memoria, Trabajo de Final de Grado/Máster
\newpage

\pagestyle{fancy}
\renewcommand{\chaptermark}[1]{ \markboth{#1}{}}
\renewcommand{\sectionmark}[1]{\markright{ \thesection.\ #1}}
\lhead[\fancyplain{}{\bfseries\thepage}]{\fancyplain{}{\bfseries\rightmark}}
\rhead[\fancyplain{}{\bfseries\leftmark}]{\fancyplain{}{\bfseries\thepage}}
\cfoot{}

% indice
\cleardoublepage
\phantomsection
\addcontentsline{toc}{chapter}{Índice}
\tableofcontents
% listado de figuras
\cleardoublepage
\phantomsection
\addcontentsline{toc}{chapter}{Llistado de Figuras}
\listoffigures
% listado de tablas
\cleardoublepage
\phantomsection
\addcontentsline{toc}{chapter}{Listado de Tablas}
\listoftables

\thispagestyle{empty}

\pagenumbering{arabic}

\pagestyle{fancy}
\renewcommand{\chaptermark}[1]{ \markboth{#1}{}}
\renewcommand{\sectionmark}[1]{\markright{ \thesection.\ #1}}
\lhead[\fancyplain{}{\bfseries\thepage}]{\fancyplain{}{\bfseries\rightmark}}
\rhead[\fancyplain{}{\bfseries\leftmark}]{\fancyplain{}{\bfseries\thepage}}
\cfoot{}

\onehalfspacing

% capitulos del documento
\chapter{Introducción}
\label{chapter:introduccion}


%%% SECTION
\section{Descripción general del problema}

En la actualidad, los procesos de minería de datos requieren grandes cantidades de datos, que en muchas ocasiones contienen información personal y privada de usuarios o personas. Aunque se realicen procesos básicos de anonimización sobre los datos, es decir, eliminación de los nombres u otros identificadores clave, existen multitud de técnicas de re-identificación que permiten volver a identificar a un usuario dentro de este conjunto de datos. En la Figura \ref{fig:context-anoni1} se presenta un mapa donde es posible contextualizar los procesos de anonimización y re-identificación dentro de un proceso de minería de datos.

\begin{figure}
	\centering
	\includegraphics[width=0.6\textwidth]{figs/image1.png}
	\caption{Pie de la imagen.}
	\label{fig:context-anoni1}
\end{figure}

\subsection{Ejemplo de subsection}

Aunque se han realizado importantes avances en preservación de la privacidad en publicación de datos, tales como el modelo \textit{k}-anonymity \cite{Sweeney:2002}.

Un ejemplo de pseudo-código se puede encontrar en el Código \ref{code:RandomSwitch-1}

\begin{algorithm}
	\caption{Pseudocódigo del algoritmo \textit{Random Switch}}
	\label{code:RandomSwitch-1}
	\begin{algorithmic}
		\REQUIRE{El grafo original $G$ y el porcentaje de anonimización $p$ que se desea aplicar.}
		\ENSURE{El grafo $G$ anonimizado.}
		\STATE $num = round(G.num\_edges() * p)$
		\STATE $i = 0$
		\WHILE {$i < num$}
		\STATE {$e_{1} = G.random\_edge()$}
		\STATE $e_{2} = G.random\_edge()$
		\STATE $new\_e_{1} = (e_{1}.origen, e_{2}.origen)$
		\STATE $new\_e_{2} = (e_{1}.destino, e_{2}.destino)$
		\IF {$!G.exist(new\_e_{1})$ \AND $!G.exist(new\_e_{2})$}
		\STATE $G.add\_edge(new\_e_{1})$
		\STATE $G.add\_edge(new\_e_{2})$
		\STATE $G.delete\_edge(e_{1})$
		\STATE $G.delete\_edge(e_{2})$
		\STATE $i=i+1$
		\ENDIF
		\ENDWHILE
		\RETURN $G$
	\end{algorithmic}
\end{algorithm}

Un ejemplo de tabla se puede ver en la Tabla \ref{table:ejemplo_vertex_refi_query}

\begin{table}
	\centering{}
	\begin{tabular}{ l || c | c | l }
		\hline
		Node ID & $\mathcal{H}_{0}$ & $\mathcal{H}_{1}$ & $\mathcal{H}_{2}$ \\
		\hline
		\hline
		Alice & $\epsilon$ & 1 & \{4\}  \\
		\hline
		Bob & $\epsilon$ & 4 & \{1, 1, 4, 4\}  \\
		\hline
		Carol & $\epsilon$ & 1 & \{4\}  \\
		\hline
		Dave & $\epsilon$ & 4 & \{2, 4, 4, 4\}  \\
		\hline
		Ed & $\epsilon$ & 4 & \{2, 4, 4, 4\}  \\
		\hline
		Fred & $\epsilon$ & 2 & \{4, 4\}  \\
		\hline
		Greg & $\epsilon$ & 4 & \{2, 2, 4, 4\}  \\
		\hline
		Harry & $\epsilon$ & 2 & \{4, 4\}  \\
		\hline
	\end{tabular}
	\caption{\textit{Vertex refinement queries}.}
	\label{table:ejemplo_vertex_refi_query}
\end{table}

% bibliografia
\addcontentsline{toc}{chapter}{Bibliografía}
\bibliographystyle{plain}
\bibliography{referencias}

\end{document}